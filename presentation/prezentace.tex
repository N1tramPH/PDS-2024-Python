\documentclass{beamer}
\usepackage{hyperref}
\usepackage[czech]{babel}

%\usetheme{Boadilla}
\usetheme{default}
\usecolortheme{seahorse}
\definecolor{myblue}{rgb}{0.2, 0.2, 0.6}
\setbeamertemplate{caption}{{\color{myblue}Obrázek:} \raggedright\insertcaption\par}
\setbeamertemplate{footline}[frame number] 

\newcommand{\authorname}{Dluhoš Matěj, Gaja Jan, Gajdošík Petr, Jurásek Petr, Phan Thanh Tú, Jan Rádl}
\newcommand{\authorsshort}{Dluhoš, Gaja, Gajdošík, Jurásek, Phan, Rádl}
\newcommand{\thesisname}{PDS: Python}

\title{\thesisname}
\author{\authorname}
\institute{Univerzita Palackého v Olomouci}
\date{\today}

\setbeamertemplate{navigation symbols}{}
\setbeamertemplate{headline}{}
\setbeamertemplate{footline}{
  \leavevmode%
  \hbox{%
  \begin{beamercolorbox}[wd=.4\paperwidth,ht=2.25ex,dp=1ex,center]{author in head/foot}%
    \usebeamerfont{author in head/foot}\authorsshort
  \end{beamercolorbox}%
  \begin{beamercolorbox}[wd=.3\paperwidth,ht=2.25ex,dp=1ex,center]{title in head/foot}%
    \usebeamerfont{title in head/foot}\thesisname
  \end{beamercolorbox}%
  \begin{beamercolorbox}[wd=.3\paperwidth,ht=2.25ex,dp=1ex,right]{date in head/foot}%
    \usebeamerfont{date in head/foot}\insertshortdate{}\hspace*{2em}
    \insertframenumber{} / \inserttotalframenumber\hspace*{2ex} 
  \end{beamercolorbox}}%
  \vskip0pt%
}

%\logo{\includegraphics[height=1cm]{UP_logo.png}}
\usepackage{tikz}
\addtobeamertemplate{headline}{}{%
    \begin{tikzpicture}[overlay, remember picture]
    	\ifnum\insertframenumber>1
        	\node[anchor=north east, inner sep=5pt] at (current page.north east) {\includegraphics[height=1cm]{obrazky/UP_logo.png}};
        \fi
    \end{tikzpicture}
}

\setbeamercolor{titlelike}{bg=,fg=}

\begin{document}

\begin{frame}
	\titlepage
\end{frame}

\begin{frame}{Obsah}
	\tableofcontents
\end{frame}

\section{Úvod do Pythonu s paralelismem}
\subsection{Historie a zaměření}
\begin{frame}{Historie a zaměření}
	\begin{itemize}
		\item Vytvořen \textbf{Guidem van Rossumem} v roce \textbf{1991}.
	\end{itemize}
	\begin{itemize}
		\item Jednoduchost a čitelnost kódu.
		\item Vysoká produktivita programátorů.
		\item Univerzální použití: od webových aplikací přes vědecké výpočty až po strojové učení.
	\end{itemize}
\end{frame}

\subsection{Paralelismus v Pythonu}
\begin{frame}{Paralelismus v Pythonu}
	\begin{itemize}
		\item Python získal základní podporu paralelismu prostřednictvím vláken ve verzi 1.5.2 (1999).
		\item Zaveden \textbf{GIL} (Global Interpreter Lock)
		\begin{itemize}
			\item [\textendash] Usnadnilo integraci nativního kódu v C.
			\item [\textendash] Vyřešilo problémy s přístupem ke sdíleným datům ve více vláknech.
			\item [\textendash] Optimalizovalo výkon na systémech s jedním jádrem.
			\item [\textendash] Omezuje paralelní výkon na vícejádrových procesorech.
			\item [\textendash] Vede k neefektivnímu využití vláken u \textbf{CPU-bound} úloh.
			\item [\textendash] \hypersetup{urlcolor=blue} \href{https://peps.python.org/pep-0703/}{PEP 703} - Making the Global Interpreter Lock Optional in CPython 
		\end{itemize}
	\end{itemize}
\end{frame}

\subsection{Proč odstranit GIL?}
\begin{frame}{Proč odstranit GIL?}
	\begin{itemize}
		\item Většina zařízení dnes obsahuje vícejádrové procesory, kde GIL představuje významné omezení.
		\item Efektivnější paralelní výpočty, například v oblastech umělé inteligence nebo numerických simulacích.
		\item Výzvy při odstranění GIL:
		\begin{itemize}
			\item [\textendash] Potřeba bezpečného a efektivního spravování sdílené paměti (například více zamykání nebo jiných synchronizačních mechanismů).
			\item [\textendash] Zvýšení složitosti implementace interpretu Pythonu.
		\end{itemize}
	\end{itemize}
\end{frame}

\subsection{Komunikační model Pythonu}
\begin{frame}{Komunikační model Pythonu}

\end{frame}

\section{Synchronizační nástroje standardní knihovny}
\subsection{Knihovna threading}
\begin{frame}{Knihovna threading}


\end{frame}

\subsection{Knihovna multiprocessing}
\begin{frame}{Knihovna multiprocessing}

\end{frame}

\subsection{Knihovna concurrent.futures}
\begin{frame}{Knihovna concurrent.futures}

\end{frame}

\subsection{Knihovna asyncio}
\begin{frame}{Knihovna asyncio}

\end{frame}


\section{Poznámka: výkonnost jazyka Python}
\begin{frame}{Poznámka: výkonnost jazyka Python}
	\begin{itemize}
    \item Nejde jazyk Python naproti samotné podstatě paralelizace?
    \begin{itemize}
      \item [\textendash] Známý pro svou nevýkonnost (daň za velkou abstrakci)      
      \item [\textendash] GIL (dnes již volitelný)
    \end{itemize}

    \item I přes omezení GIL má paralelizace opodstatnění, pokud musí vlákna často čekat na externí události (I/O)
    \begin{itemize}
      \item obsluha klienta, databáze...
      \item framework \textbf{asyncio} (asynchronní programování)
    \end{itemize}
  \end{itemize}
\end{frame}


\section{Řešení problémů paralelizace a výkonu v Pythonu}
\begin{frame}{Řešení problémů paralelizace a výkonu v Pythonu}
  \begin{enumerate}
    \item Použití Multiprocessing 
    \begin{itemize}
      \item [\textendash] Poněkud těžkopadné
      \item [\textendash] Náročná správa komunikace (nesdílejí paměť)
    \end{itemize}

    \item Použití jiné implementace Pythonu s JIT kompilací
    \begin{itemize}
      \item [\textendash] Dynamická kompilace do strojového kódu, optimalizace
      \item [\textendash] \textbf{Numba}, PyPy, Jyphon (JVM), IronPython (CLR)\dots
      \item [\textendash] Některé nejsou omezeny GIL
    \end{itemize}

    \item Statická kompilace do nižšího jazyka
    \begin{itemize}
      \item [\textendash] \href{https://cython.org/}{Cython}, \href{https://github.com/mypyc/mypyc}{mypyc}\dots
    \end{itemize}

    \item Použití externích knihoven
    \begin{itemize}
      \item [\textendash] Python slouží jako jakási vysokoúrovňová "fasáda"
      \item [\textendash] Pro "CPU-bound" operace se volají metody implementované v nižších jazycích (C, C++...)
      \item [\textendash] Tyto jazyky nejsou omezeny GIL, mohou (volně) využívat paralelizace
    \end{itemize}

    \item Nepoužívat Python
  \end{enumerate}
\end{frame}


\section{Nástroje externích knihoven}
\begin{frame}{Nástroje externích knihoven}
	\begin{itemize}
    \item Dask
      \begin{itemize}
        \item [\textendash] Navržena pro \textbf{paralelizovaný} i \textbf{distribuovaný výpočet} na rozsáhlých datech

        \item [\textendash] Rozděluje úlohy do menších částí (\textit{tasks}) a vytváří graf závislostí, kterým se řídí následný výpočet

        \item [\textendash] \textbf{Threaded-, Multiprocessing-, Distributed-} scheduler

        \item [\textendash] Využívá dalších knihoven jako jsou NumPy, Pandas a dalších pro rychlejší nízkoúrovňové výpočty
      \end{itemize}
    \item Další:
      \begin{itemize}
        \item [\textendash] Ray
        \item [\textendash] Celery\dots
      \end{itemize}
  \end{itemize}
\end{frame}


\section{Nástroje pro distribuované systémy}
\subsection{Knihovna socket}
\begin{frame}{Knihovna socket}

\end{frame}

\subsection{Knihovna http.server}
\begin{frame}{Knihovna http.server}

\end{frame}

\subsection{RabbitMQ}
\begin{frame}{RabbitMQ}

\end{frame}

\subsection{Další nástroje}
\begin{frame}{Další nástroje}

\end{frame}

\end{document}
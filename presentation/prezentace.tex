\documentclass{beamer}
\usepackage{hyperref}
\usepackage[czech]{babel}

%\usetheme{Boadilla}
\usetheme{default}
\usecolortheme{seahorse}
\definecolor{myblue}{rgb}{0.2, 0.2, 0.6}
\setbeamertemplate{caption}{{\color{myblue}Obrázek:} \raggedright\insertcaption\par}
\setbeamertemplate{footline}[frame number] 

\newcommand{\authorname}{Dluhoš Matěj, Gaja Jan, Gajdošík Petr, Jurásek Petr, Phan Thanh Tú, Rádl Jan}
\newcommand{\authorsshort}{Dluhoš, Gaja, Gajdošík, Jurásek, Phan, Rádl}
\newcommand{\thesisname}{PDS: Python}

\title{\thesisname}
\author{\authorname}
\institute{Univerzita Palackého v Olomouci}
\date{\today}

\setbeamertemplate{navigation symbols}{}
\setbeamertemplate{headline}{}
\setbeamertemplate{footline}{
  \leavevmode%
  \hbox{%
  \begin{beamercolorbox}[wd=.4\paperwidth,ht=2.25ex,dp=1ex,center]{author in head/foot}%
    \usebeamerfont{author in head/foot}\authorsshort
  \end{beamercolorbox}%
  \begin{beamercolorbox}[wd=.3\paperwidth,ht=2.25ex,dp=1ex,center]{title in head/foot}%
    \usebeamerfont{title in head/foot}\thesisname
  \end{beamercolorbox}%
  \begin{beamercolorbox}[wd=.3\paperwidth,ht=2.25ex,dp=1ex,right]{date in head/foot}%
    \usebeamerfont{date in head/foot}\insertshortdate{}\hspace*{2em}
    \insertframenumber{} / \inserttotalframenumber\hspace*{2ex} 
  \end{beamercolorbox}}%
  \vskip0pt%
}

%\logo{\includegraphics[height=1cm]{UP_logo.png}}
\usepackage{tikz}
\addtobeamertemplate{headline}{}{%
    \begin{tikzpicture}[overlay, remember picture]
    	\ifnum\insertframenumber>1
        	\node[anchor=north east, inner sep=5pt] at (current page.north east) {\includegraphics[height=1cm]{obrazky/UP_logo.png}};
        \fi
    \end{tikzpicture}
}

\setbeamercolor{titlelike}{bg=,fg=}

\begin{document}

\begin{frame}
	\titlepage
\end{frame}

\begin{frame}{Obsah}
	\tableofcontents
\end{frame}

\section{Úvod do Pythonu s paralelismem}
\subsection{Historie a zaměření}
\begin{frame}{Historie a zaměření}
    \begin{columns}
        \column{0.55\textwidth}
        \begin{itemize}
            \item Vytvořen \textbf{Guidem van Rossumem} v roce \textbf{1991}.
            \item Jednoduchost a čitelnost kódu.
            \item Vysoká produktivita programátorů.
            \item Univerzální použití: od webových aplikací přes vědecké výpočty až po strojové učení.
            \item Python získal základní podporu paralelismu prostřednictvím vláken ve verzi 1.5.2 (1999).
        \end{itemize}
        
        \column{0.45\textwidth}
        \begin{figure}
        		\includegraphics[width=0.75\textwidth]{obrazky/Guido_van_Rossum.jpg}
        		\caption{Guido van Rossum}
      		\end{figure}
    \end{columns}
\end{frame}

\subsection{GIL - Global Interpreter Lock}
\begin{frame}{Co to je GIL a proč je potřeba?}
	\textbf{Co to je GIL?}
	\begin{itemize}
		\item GIL je mutex umožňující pouze jednomu vláknu mít kontrolu nad Python interpreterem.
	\end{itemize}
	\textbf{Proč je potřeba?}
	\begin{itemize}
		\item Správa paměti v CPythonu není bezpečná pro více vláken.
		\item Python využívá počítání referencí.
		\item Problém nastane, když dvě vlákna současně zvyšují nebo snižují referenci.
		\item To vede k úniku nebo k nesprávnému uvolnění paměti.
	\end{itemize}
\end{frame}

\subsection{Rozdíl vláken a procesů}
\begin{frame}{Vlákno}
	\begin{itemize}
		\item Entita v rámci procesu.
		\item \textbf{Klíčová fakta:}
			\begin{itemize}
				\item [\textendash] V rámci jednoho procesu může být spuštěno více vláken.
				\item [\textendash] Paměť je sdílena mezi všemi vlákny.
				\item [\textendash] Spuštění vlákna je rychlejší než spuštění procesu.
				\item [\textendash] Skvělé pro úlohy závislé na I/O.
				\item [\textendash] Lehká paměťová náročnost.
			\end{itemize}
		\item \textbf{Nevýhody:}
			\begin{itemize}
				\item [\textendash] Jedno GIL pro všechna vlákna.
				\item [\textendash] Multithreading nemá efekt u úloh náročných na CPU.
				\item [\textendash] Vlákna nelze přerušit a ukončit.
				\item [\textendash] Zvýšený potenciál vzniku chyby souběhu.
			\end{itemize}
	\end{itemize}
\end{frame}

\begin{frame}{Proces}
	\begin{itemize}
		\item Instance programu.
		\item \textbf{Klíčová fakta:}
			\begin{itemize}
				\item [\textendash] Využívá více jader a procesorů.
				\item [\textendash] Nový proces se spouští nezávisle na prvním procesu.
				\item [\textendash] Má oddělený paměťový prostor.
				\item [\textendash] Každý proces má svůj vlastní GIL.
				\item [\textendash] Ideální pro úlohy náročné na CPU.
				\item [\textendash] Procesy lze přerušit nebo ukončit.
			\end{itemize}
		\item \textbf{Nevýhody:}
			\begin{itemize}
				\item [\textendash] Spuštění procesu je pomalejší než spuštění vlákna.
				\item [\textendash] Vyšší paměťová náročnost.
				\item [\textendash] Složitější IPC.
			\end{itemize}
	\end{itemize}
\end{frame}

\begin{frame}{Potřebujeme vůbec GIL?}
	\textbf{Je tedy potřeba?}
	\begin{itemize}
		\item V brzkých dnech Pythonu usnadnil integraci nativního kódu v C.
		\item Vyřešil problémy s přístupem ke sdíleným datům.
		\item Optimalizoval výkon na systémech s jedním jádrem.
	\end{itemize}
	\textbf{Ale...}
	\begin{itemize}
		\item Omezuje paralelní výkon na vícejádrových procesorech.
		\item Vede k neefektivnímu využití vláken u CPU-bound úloh.
	\end{itemize}
	\textbf{\hypersetup{urlcolor=blue} \href{https://peps.python.org/pep-0703/}{PEP 703}} - Making the Global Interpreter Lock Optional in CPython.
\end{frame}


%\subsection{Paralelismus v Pythonu}
%\begin{frame}{Paralelismus v Pythonu}
%	\begin{itemize}
%		\item Python získal základní podporu paralelismu prostřednictvím vláken ve verzi 1.5.2 (1999).
%		\item Zaveden \textbf{GIL} (Global Interpreter Lock)
%		\begin{itemize}
%			\item [\textendash] Usnadnilo integraci nativního kódu v C.
%			\item [\textendash] Vyřešilo problémy s přístupem ke sdíleným datům ve více vláknech.
%			\item [\textendash] Optimalizovalo výkon na systémech s jedním jádrem.
%			\item [\textendash] Omezuje paralelní výkon na vícejádrových procesorech.
%			\item [\textendash] Vede k neefektivnímu využití vláken u \textbf{CPU-bound} úloh.
%			\item [\textendash] \hypersetup{urlcolor=blue} \href{https://peps.python.org/pep-0703/}{PEP 703} - Making the Global Interpreter Lock Optional in CPython 
%		\end{itemize}
%	\end{itemize}
%\end{frame}

%\subsection{Proč odstranit GIL?}
%\begin{frame}{Proč odstranit GIL?}
%	\begin{itemize}
%		\item Většina zařízení dnes obsahuje vícejádrové procesory, kde GIL představuje významné omezení.
%		\item Efektivnější paralelní výpočty, například v oblastech umělé inteligence nebo numerických simulacích.
%		\item Výzvy při odstranění GIL:
%		\begin{itemize}
%			\item [\textendash] Potřeba bezpečného a efektivního spravování sdílené paměti (například více zamykání nebo jiných synchronizačních mechanismů).
%			\item [\textendash] Zvýšení složitosti implementace interpretu Pythonu.
%		\end{itemize}
%	\end{itemize}
%\end{frame}


\section{Synchronizační nástroje standardní knihovny}
\subsection{Knihovna threading}

\begin{frame}{Knihovna threading}
    \begin{itemize}
        \item Vestavený modul pro paralelní spouštění vláken
        \item Omezena GIL
        \item Vlákna využívají stejný paměťový prostor
        \vskip 0.35in
        \item Často využíváné pro:
        \begin{itemize}
            \item I/O operace (Read/Write, Síťování, user input)
            \item Grafické uživatelské rozhraní 
        \end{itemize}
    \end{itemize}
\end{frame}

\begin{frame}[fragile]{Třída Thread}
    \begin{itemize}
        \item Instance reprezentuje vlákno
        \item Předání funkce a argumentů
        \vskip 0.2in

        \item \textit{start()} - spustí vytvořené vlákno
        \item \textit{join()} – Počká na dokončení práce vlákna
        \item \textit{is\_alive()} – Pracuje stále vlákno?
        \item \textit{local()} – data konkrétního vlákna

        \vskip 0.2in
        \item Při práci s vlákny může dojít k synchronizačním problémům
        \item Modul poskytuje mnoho synchronizačních nástrojů
    \end{itemize}
    \scriptsize
    \begin{semiverbatim}
    t = threading.Thread(target=some_function, args=(some_arg,))
    t.start()
    # Nějaký kód
    t.join()
    \end{semiverbatim}
\end{frame}

\begin{frame}[fragile]{Lock}
    \begin{itemize}
            \item Zajistí, že ke sdíleným zdrojům bude mít přítup pouze jedno vlákno
            \item Můžeme zamknou příkaz nebo skupinu příkazu

            \vskip 0.15in
            \item \textit{acquire()} – zamkne objekt
            \item \textit{release()} – odemkne objekt
            \item klíčové slovo with - zamkne/odemkne blok automaticky

            \vskip 0.15in
            \item Varianta \textbf{RLock} pro rekurzivní funkce
    \end{itemize}
    \scriptsize
    \begin{semiverbatim}
    lock = threading.Lock()

    def function():
	    # Nějaký kód
	    with lock:  
	        # Další kód
    \end{semiverbatim}
\end{frame}

\begin{frame}[fragile]{Semaphore}
    \begin{itemize}
        \item Dokáže omezit počet vláken, které mají přístup ke zdroji
        \item Konstruktor očekává value, určuje maximální počet přístupů ke zdroji
        \newline (Varianta Binary Semaphore – maximálně 1 přístup)

        \vskip 0.25in
        \item \textit{acquire()} – zjistí jestli je count $>$ 0, pokud ano, dekrementuje count, jinak zablokuje vlákno
        \item \textit{release()} – inkrementuje count, umožní tak dalšímu vláknu přístup
        \item Opět with statement dělá automaticky
    \end{itemize}
    \scriptsize
    \begin{semiverbatim}
    sem = threading.Semaphore(limit)

    def function():
        # Nějaký kód
        with sem:
            # kritická sekce
    \end{semiverbatim}
\end{frame}

\begin{frame}[fragile]{Barrier}
    \begin{itemize}
        \item Umožňuje zablokovat vlákna, dokud všechna nedojdou na určité místo v programu
        \item Až všechna vlákna dosáhnou bariéry, pokračují dál v práci
        \item Vhodné pokud vlákna pracují ve fázích, kdy po dokončení jedné se spustí další

        \vskip 0.25in
        \item \textit{wait()} – Místo, kde se vlákno zastaví, dokud ostatní vlákna k němu nedojdou
    \end{itemize}
    \scriptsize
    \begin{semiverbatim}
    barrier = threading.Barrier(limit)

    def function():
        # Nějaký kód
        barrier.wait()
        # Další kód
    \end{semiverbatim}
\end{frame}

\begin{frame}{Event}
    \begin{itemize}
        \item Jednoduchý mechanizmus, který umožňuje signalizaci mezi vlákny
        \item Používá interní příznak, který nabývá hodnoty True nebo False
        \vskip 0.25in

        \item \textit{wait()} – vlákna budou blokována, dokud nebude flag nastaven na True
        \item \textit{set()} – Nastaví flag na True, vlákna volající wait() budou propuštěna
        \item \textit{clear()} – Nastaví flag na False, vlákna volající wait() budou zablokována
    \end{itemize}
\end{frame}

\begin{frame}[fragile]{Event}
    \begin{semiverbatim}
    event = threading.Event()

    def function1():
    	# Nějaký kód
    	event.wait()
    	# Další kód
    
    def function2():
    	# Nějaký kód
    	event.set()
    	# Další kód  
    \end{semiverbatim}
\end{frame}

\begin{frame}[fragile]{Condition}
    \begin{itemize}
        \item Funguje obdobně, ale používá se současně se zámkem

        \vskip 0.25in
        \item \textit{wait()} – stejné, ale navíc odemkne zámek
        \item \textit{notify()} – odblokuje jedno čekající vlákno
        \item \textit{notify\_all()} – odblokuje všechna čekající vlákna
    \end{itemize}
    \scriptsize
    \begin{semiverbatim}
    cond = threading.Condition()

    def function1():
    	with cond:
    		# Nějaký kód
    		cond.wait()
    		# Další kód
    
    def function2():
    	with cond:
    		# Nějaký kód
    		cond.notify()
    		# Další kód
    \end{semiverbatim}
\end{frame}

\subsection{Knihovna multiprocessing}
\begin{frame}{Knihovna multiprocessing}
    \begin{itemize}
        \item Modul pro paralelizaci procesů
        \item Nepracuje s vlákny ale s procesy
        \item Umožňuje překonat limitace GIL
        \item Vhodné pro složitější výpočty
        
        \vskip 0.3in
        \item Výhody:
        \begin{itemize}
            \item Každý vytvořený proces běží nezávisle a může být vykonáván na jiných jádrech procesoru
            \item Všechny procesy mají vlastní paměť
            \item Vestavené metody pro bezpečnou práci se sdílenými prostředky
        \end{itemize}
    \end{itemize}
\end{frame}

\begin{frame}[fragile]{Třída Proces}
    \begin{itemize}
        \item Instance reprezentuje jeden nezávislý proces
        \item Tvorba stejná jako u threading
        \item Stejné funkce
        \vskip 0.25in

        \item Využití Pool
        \begin{itemize}
            \item Umožňuje efektivně spravovat procesy
            \item Automatizuje jejich tvorbu a rozdělení mezi jádra CPU
            \item \textit{map()} - pro paralelní provedení funkce na každý prvek z kolekce dat
        \end{itemize}
    \end{itemize}
    \scriptsize
    \begin{semiverbatim}
    pool = Pool(processes=num_of_processes) 
    result = pool.map(some_function, iterable_data)
    \end{semiverbatim}  
\end{frame}

\begin{frame}{Data a komunikace}
    \begin{itemize}
        \item Sdílení dat
        \begin{itemize}
            \item Modul poskytuje objekty Array a Value pro sdílení dat mezi procesy
        \end{itemize}

        \item Komunikace
        \begin{itemize}
            \item Pro plné využití síly procesů je potřeba komunikace mezi nimi
            \item Objekty Queue a Pipe pro zasílání zpráv
            \item S Queue se pracuje stejně jako s klasickou frontou
            \item Pipe je vhodnější pro komunikaci mezi pouze dvěma procesy
        \end{itemize}
        \item Synchronizace
        \begin{itemize}
            \item Stejné nástroje pro synchronizaci jako v modulu threading
            \item Lock, Semaphore, Barrier, Event, Condition
        \end{itemize}
    \end{itemize}
\end{frame}

\begin{frame}[fragile]{Data a komunikace}
    \scriptsize
    \begin{semiverbatim}
    value = Value('i', init_value)  
    result = Array('i', size)
    
    def example_function(data,value,result):  
        for i, num in range(5):  
            result[i] = num * num  
            value.value += num

    q = multiprocessing.Queue()
    conn1, conn2 = multiprocessing.Pipe()

    def producer(conn,q):
        q.put("Queue Message")
        conn.send("Pipe Message")
        conn.close()

    def consumer(conn,q):
        msg = q.get()
        print(msg)
        msg = conn.recv()
        print(msg)
        conn.close()
    \end{semiverbatim}
\end{frame}

\subsection{Knihovna asyncio}
\begin{frame}{Knihovna asyncio}
    \begin{itemize}
        \item Umožňuje asynchronní zpracování vstupně-výstupních operací
        \item Založeno na kooperativním multitaskingu
        \item Můžeme se tak vyhnout nečinnosti při čekání na dokončení I/O operací
        \item Vhodné pro síťové operace, práce s databází, čtení souborů…
        
        \vskip 0.35in
        \item \textbf{Event Loop} – koordinuje spuštění asynchronních operací
        \item \textbf{Coroutine} – speciální asynchronní funkce
        \item \textbf{Task} – zajišťuje běh coroutine v event loopu
        \item \textbf{Future} – Předběžný výsledek operace, která nebyla ještě dokončena
    \end{itemize}
\end{frame}

\begin{frame}{Třída Task}
    \begin{itemize}
        \item Při vytváření Tasku předáváme korutinu
        \item Následně je její běh naplánován Event Loopem
        \item Metody pro práci s Tasky během toho co se vykonávají
        \begin{itemize}
            \item \textit{.result()}
            \item \textit{.exception()}
            \item \textit{.cancel()}
        \end{itemize}
        
        \vskip 0.25in
        \item Další užitečné nástroje
        \begin{itemize}
            \item Stejné synchronizační nástroje jako v předchozích modulech

            \item \textit{.wait\_for()} umožňuje nastavit maximální čas čekání na dokončení. Vyvolává TimeoutError
        \end{itemize}
    \end{itemize}
\end{frame}

\begin{frame}[fragile]{Třída Task}
    \scriptsize
    \begin{semiverbatim}
    async def coroutine(): 
    	do_something()
    	# Simulace nějaké I/O operace, na kterou je potřeba počkat
    	await asyncio.sleep(1) 

    async def main():
        task = asyncio.create_task(coroutine(arg))
        await task

    asyncio.run(main())
    \end{semiverbatim}
\end{frame}


\section{Poznámka: výkonnost jazyka Python}
\begin{frame}{Poznámka: výkonnost jazyka Python}
	\begin{itemize}
    \item Nejde jazyk Python naproti samotné podstatě paralelizace?
    \begin{itemize}
      \item [\textendash] Známý pro svou nevýkonnost (daň za velkou abstrakci)      
      \item [\textendash] GIL (dnes již volitelný)
    \end{itemize}

    \item I přes omezení GIL má paralelizace opodstatnění, pokud musí vlákna často čekat na externí události (I/O)
    \begin{itemize}
      \item obsluha klienta, databáze...
      \item framework \textbf{asyncio} (asynchronní programování)
    \end{itemize}
  \end{itemize}
\end{frame}


\section{Řešení problémů paralelizace a výkonu v Pythonu}
\begin{frame}{Řešení problémů paralelizace a výkonu v Pythonu}
  \begin{enumerate}
    \item Použití Multiprocessing 
    \begin{itemize}
      \item [\textendash] Poněkud těžkopadné
      \item [\textendash] Náročná správa komunikace (nesdílejí paměť)
    \end{itemize}

    \item Použití jiné implementace Pythonu s JIT kompilací
    \begin{itemize}
      \item [\textendash] Dynamická kompilace do strojového kódu, optimalizace
      \item [\textendash] \textbf{Numba}, PyPy, Jyphon (JVM), IronPython (CLR)\dots
      \item [\textendash] Některé nejsou omezeny GIL
    \end{itemize}

    \item Statická kompilace do nižšího jazyka
    \begin{itemize}
      \item [\textendash] \href{https://cython.org/}{Cython}, \href{https://github.com/mypyc/mypyc}{mypyc}\dots
    \end{itemize}

    \item Použití externích knihoven
    \begin{itemize}
      \item [\textendash] Python slouží jako jakási vysokoúrovňová "fasáda"
      \item [\textendash] Pro "CPU-bound" operace se volají metody implementované v nižších jazycích (C, C++...)
      \item [\textendash] Tyto jazyky nejsou omezeny GIL, mohou (volně) využívat paralelizace
    \end{itemize}

    \item Nepoužívat Python
  \end{enumerate}
\end{frame}


\section{Nástroje externích knihoven}
\begin{frame}{Nástroje externích knihoven}
	\begin{itemize}
    \item Dask
      \begin{itemize}
        \item [\textendash] Navržena pro \textbf{paralelizovaný} i \textbf{distribuovaný výpočet} na rozsáhlých datech

        \item [\textendash] Rozděluje úlohy do menších částí (\textit{tasks}) a vytváří graf závislostí, kterým se řídí následný výpočet

        \item [\textendash] \textbf{Threaded-, Multiprocessing-, Distributed-} scheduler

        \item [\textendash] Využívá dalších knihoven jako jsou NumPy, Pandas a dalších pro rychlejší nízkoúrovňové výpočty
      \end{itemize}
    \item Další:
      \begin{itemize}
        \item [\textendash] Ray
        \item [\textendash] Celery\dots
      \end{itemize}
  \end{itemize}
\end{frame}


\section{Nástroje pro distribuované systémy}
\subsection{Knihovna socket}
\begin{frame}{Knihovna socket}
	\begin{itemize}
		\item Modul \texttt{socket} je základní nástroj pro implementaci síťové komunikace.
		\item Je \textbf{nízkoúrovňový} – jsou na něm založeny všechny ostatní moduly Pythonu vyšší úrovně pro propojení v síti.
		\item Podporuje většinu běžných protokolů, včetně TCP a UDP.
		\item \textbf{Výhody:}
		\begin{itemize}
			\item Přímá kontrola nad socketovou komunikací.
			\item Flexibilita pro implementaci vlastních protokolů.
		\end{itemize}
		\item \textbf{Nevýhody:}
		\begin{itemize}
			\item Vyžaduje znalosti síťových protokolů.
			\item Implementace složitější logiky (např. HTTPS) je náročnější.
		\end{itemize}
	\end{itemize}
\end{frame}

\begin{frame}{Knihovna socket - tvorba socketu}
	\begin{itemize}
        \begin{semiverbatim}
		import socket

		socket = socket.socket(
			family=AF_INET,
			type=SOCK_STREAM
		)
        \end{semiverbatim}

		\item \texttt{family}  - určuje, jaký protokol se použije na síťové vrstvě.
		\begin{itemize}
			\item \texttt{socket.AF\_INET} - pro IPv4 adresy.
			\item \texttt{socket.AF\_INET6} - pro IPv6 adresy.
			\item \texttt{socket.AF\_UNIX} - Unixový socket (pro lokální komunikaci mezi procesy).
		\end{itemize}
		\item \texttt{type} - určuje, jaký protokol se použije na transportní vrstvě.
		\begin{itemize}
			\item \texttt{socket.SOCK\_STREAM} - používá TCP protokol.
			\item \texttt{socket.SOCK\_DGRAM} - používá UDP protokol.
		\end{itemize}
	\end{itemize}
\end{frame}

\begin{frame}{Knihovna socket - odeslání zprávy}
    \begin{semiverbatim}
	s.connect(('localhost', 12345)) # navázání spojení
	s.sendall(b'Ahoj!') #odešle všechna data
	s.close() # ukončí socket s
    \end{semiverbatim}
\end{frame}

\begin{frame}{Knihovna socket - poslouchání  a přijímání dat}
    \begin{semiverbatim}
	addr = ('localhost', 12345) # adresa na které posloucháme
	s.bind(addr) # přiřadí socket k adrese
	s.listen() # umožňuje přijímat příchozí připojení
	conn, addr = s.accept() # čeká se na příchozí připojení
	print(f'Připojeno: {addr}')
	data = conn.recv(1024) #přijme data, max. 1024B
	conn.sendall(b'Ahoj!') # pošle odpověď
	conn.close() # ukončí socket conn
	s.close() # ukončí socket s
    \end{semiverbatim}
\end{frame}

\subsection{Knihovna http.server}
\begin{frame}{Knihovna http}
	\begin{itemize}
		\item Modul \texttt{http} v Pythonu poskytuje nástroje pro práci s protokolem HTTP.
		\item Je rozdělen na několik podmodulů:
		\begin{itemize}
			\item Mezi ně patří např.: \texttt{http.client}, \texttt{http.server}, \texttt{http.cookies} \ldots
		\end{itemize}
		\item Využívá ho mnoho vyšších knihoven, jako je \texttt{requests} nebo \texttt{flask}.
		\item \textbf{Výhody:}
		\begin{itemize}
			\item Vyšší úroveň abstrakce.
			\item Snazší implementace běžných HTTP funkcionalit.
		\end{itemize}
		\item \textbf{Nevýhody:}
		\begin{itemize}
			\item Omezená flexibilita ve srovnání s nízkoúrovňovou komunikací pomocí \texttt{socket}.
		\end{itemize}
	\end{itemize}	
\end{frame}
\begin{frame}{Knihovna http.server}
    \begin{semiverbatim}
	from http.server import HTTPServer, BaseHTTPRequestHandler

	class MyHandler(BaseHTTPRequestHandler):

		# Definujeme metodu pro zpracování GET požadavků
		def do_GET(self):
			pass

	server = HTTPServer(("localhost", 8000), MyHandler)
	print("Server bezi na portu 8000...")
	server.serve_forever()
    \end{semiverbatim}

	\begin{itemize}
		\item Dva typy serverů:
		\begin{itemize}
			\item \texttt{http.server.HTTPServer(args)}.
			\item \texttt{http.server.ThreadingHTTPServer(args)}.
		\end{itemize}
		\item Argumenty \texttt{args}:
		\begin{itemize}
			\item \texttt{server\_address} - dvojice IP adresa, port.
			\item \texttt{RequestHandlerClass} - zpracovává příchozí HTTP požadavky (např. GET, POST) a definuje, jak server na tyto požadavky odpoví.
		\end{itemize}
	\end{itemize}	
\end{frame}

\begin{frame}{Knihovna http.server}
    \begin{semiverbatim}
	def do_GET(self):
		# Odeslání odpovědi s kódem 200 (OK)
		self.send_response(200)
		# Nastavení hlavičky Content-Type pro HTML obsah
		self.send_header("Content-type", "text/html") 
		self.end_headers()
		# Odeslání těla odpovědi – text "Hello, World!"     
		self.wfile.write(b"Hello, World!") 
    \end{semiverbatim}
\end{frame}

\subsection{RabbitMQ}
\begin{frame}{RabbitMQ}

\end{frame}

\subsection{Další nástroje}
\begin{frame}{Další nástroje}

\end{frame}

\end{document}